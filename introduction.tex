\section{Introduction} 
  Study of intergalactic Hydrogen  in the early universe via the 21\,cm line is forecast to provide a wealth of astrophysical and cosmological information.  The 21\,cm line is both optically thin and spectrally narrow, making possible full tomographic reconstruction. Cosmological Hydrogen, which makes up 3/4 of baryonic matter, is neutral over cosmic time from recombination until reionized by the first batch of UV emitters (stars and accretion disks).  Reviews of 21 cm cosmology, astrophysics and observing can be found in \cite{Morales:2010p8093,Furlanetto:2006p2267,Pritchard:2012p9555,zaroubi2013epoch}.
  
Direct detection of HI during the Epoch of Reionization (cosmological redshifts $5<z<13$) is currently the goal of several new radio arrays. The LOw Frequency ARray \citep[LOFAR;]{Yatawatta:2013p9699}, the Donald C. Backer Precision Array for Probing the Epoch of Reionization (PAPER; \citet{Parsons:2014p10499}) and the Murchison Widefield Array (MWA; ) are all currently conducting long observing campaigns.

The analysis of the resulting data presents several challenges. The signal is faint; initial detection is being sought in the power spectrum with thousands of hours (multiple seasons) of integration required. This faint spectral line signal sits atop a continuum foreground four orders of magnitude brighter. At the same time, the instruments are the first of a new breed; fully correlated phased arrays with wide fields of view that strain the conventional assumptions of radio astronomy practice. 

The path from observation to power spectrum can be roughly divided into two parts: removal of foregrounds and estimation of power spectrum.\footnote{While statistical measures  such as \citet{Barkana:2008p2154} have also been proposed we choose the power spectrum for our initial analysis because the interferometer naturally measures in the Fourier plane.} Recently two sorts of foreground removal have been suggested. Blind filtering, such as the delay/fringe-rate filtering approach \cite{Parsons:2012p8896,Liu:2014p10462,Liu:2014p10463},  which has been applied to data from the PAPER telescope \cite{Parsons:2014p10499}, applies a small amount of knowledge about the instrument to filter modes likely to be contaminated.  This method is comparatively robust in the face of uncertainty about the instrument and the sky. Meanwhile,  full forward modeling and subtraction of sky model such as that implemented by LOFAR (see e.g. \cite{Jelic:2008p2130,Yatawatta:2013p9699}), requires a much higher fidelity model of the instrument and the sky \citep{Datta:2010p8781,Vedantham:2012p10297}. If successful, this latter approach has the benefit of opening the most sensitive power spectrum modes and substantially improving the ability of early measurements to distinguish between reionization models \cite{Beardsley:2013p9952},Pober:2014p10390}. Recent work towards the goal of foreground subtraction includes better algorithmic handling of wide field imaging effects \cite{Tasse:2012p9459,Bhatnagar..2013ApJ,Sullivan:2012p9457,Ord:2010p8442}}, continually improving catalogs of sky emission \citep[][LOFAR MSSS, and MWA GLEAM surveys]{deOliveiraCosta:2008p2242,Jacobs:2011p8438}}. Ongoing operation of the next generation low frequency arrays --LOFAR, PAPER and MWA are all in their second or third year of operation-- continues to push the refinement instrumental models and improve the accuracy of model subtraction.  

Precise foreground subtraction is only part of the picture. Estimation of the power spectrum that takes into account instrumental and sky uncertainty can also ameliorate some of the conflict between foreground and background.  Work on these fronts has focused on improving the propagation of instrumental errors and on better understanding the mathematical basis for estimating the power spectrum.  Hazelton et. al. have built a power spectrum pipeline on top of the precision instrument modeling provided by the Sullivan et al implementation of Full Holographic Deconvolution. In simulation, this analysis methodology has provided a detailed look at the true instrumental view of the foreground power spectrum \citep{Morales:2012p8790}.  Meanwhile Trott et al have developed an optimal estimator approach which aims to  More complete propagation of instrumental parameters beyond model subtraction and into the power spectrum.

On the MWA these advances are reflected in the continuing refinement of calibration and imaging software. The focus of this development effort has centered on Full Holographic Deconvolution (FHD; Sullivan et al 2014) and the MWA Real Time System (RTS; Ord et al 2010).  


Power spectrum estimation is also an active of current study with several proposed variations (Dillon, Cath, Bryna, P14) which variously require different levels of interaction with the foreground removal step.
%more review of wedge, imaging (RTS, A-Projection), power spectrum methods.




In section XXX we describe our observations strategy, in section XXX calibration, imaging and foreground removal, in section XXX power spectrum estimation, and conclude in XXX.