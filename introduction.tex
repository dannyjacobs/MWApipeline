\section{Introduction} 
  Study of intergalactic Hydrogen  in the early universe via the 21\,cm line is forecast to provide a wealth of astrophysical and cosmological information.  The 21\,cm line is both optically thin and spectrally narrow, making possible full tomographic reconstruction.  Direct detection of HI during the Epoch of Reionization (cosmological redshifts $5<z<13$) is currently the goal of several new radio arrays. The signal is faint; initial detection is being sought in the power spectrum. Reviews of 21 cm cosmology and observing can be found in \cite{Morales:2010p8093,Furlanetto:2006p2267,Pritchard:2012p9555,zaroubi2013epoch}.
  




Science case for EoR.

Overview of pipeline components and how they fit together.

In section XXX we describe our observations strategy, in section XXX calibration, imaging and foreground removal, in section XXX power spectrum estimation, and conclude in XXX.