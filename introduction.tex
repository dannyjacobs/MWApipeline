\section{Introduction} 
  Study of intergalactic Hydrogen  in the early universe via the 21\,cm line is forecast to provide a wealth of astrophysical and cosmological information.  The 21\,cm line is both optically thin and spectrally narrow, making possible full tomographic reconstruction. Cosmological Hydrogen, which makes up 3/4 of baryonic matter, is neutral over cosmic time from recombination until reionized by the first batch of UV emitters (stars and accretion disks).  Reviews of 21 cm cosmology, astrophysics and observing can be found in \cite{Morales:2010p8093,Furlanetto:2006p2267,Pritchard:2012p9555,zaroubi2013epoch}.
  
Direct detection of HI during the Epoch of Reionization (cosmological redshifts $5<z<13$) is currently the goal of several new radio arrays. The LOw Frequency ARray (), the Donald C. Backer Precision Array for Probing the Epoch of Reionization (PAPER; ) and the Murchison Widefield Array (MWA; ) are all currently conducting long observing campaigns.

The analysis of the resulting data presents several challenges. The signal is faint; initial detection is being sought in the power spectrum with thousands of hours (multiple seasons) of integration required. This faint spectral line signal sits atop a continuum foreground four orders of magnitude brighter. At the same time, the instruments are the first of a new breed; fully correlated phased arrays with wide fields of view that strain the conventional assumptions of radio astronomy imaging practice. 

The path from observation to power spectrum can be roughly divided into two parts: removal of foregrounds and estimation of power spectrum.\footnote{While statistical measures  such as \citet{Barkana:2008p2154} have also been proposed we choose the power spectrum for our initial analysis because the interferometer naturally measures in the Fourier plane.} Recently two sorts of foreground removal have been suggested. Blind filtering, which applies a small amount of knowledge about the instrument to filter modes likely to be contaminated (delay spectrum, principle component analysis) is robust in the face of uncertainty while the full forward modeling and subtraction of sky model (eg catalogs and images) requires a much higher level of instrumental and sky certainty. If successfull, this latter approach has the benefit of opening the most sensitive power spectrum modes and substantially improving the ability of early measurements to distinguish between reionization models \citep{Pober:2014p10390}.  Power spectrum estimation is also an active of current study with several proposed variations (Dillon, Cath, Bryna, P14) which variously require different levels of interaction with the foreground removal step.
%more review of wedge, imaging (RTS, A-Projection), power spectrum methods.




In section XXX we describe our observations strategy, in section XXX calibration, imaging and foreground removal, in section XXX power spectrum estimation, and conclude in XXX.