\subsection{Imager \#2: FHD}
Fast Holographic Deconvolution (FHD, \citet{an_Bunton_Cappallo_et_al__2012}) is a calibration and imaging algorithm designed for very wide field of view interferometers with direction- and antenna-dependent beam patterns. FHD has particularly been designed with a focus on producing an accurate measurement of the power spectrum complete with proper error propagation. In particular, we use the holographic beam pattern for gridding visibilities to the u-v plane, and its Hermitian conjugate for de-gridding simulations to form model visibilities. The holographic beam model is composed of the measured antenna response to the electric field for each antenna element and at every fine frequency channel, convolved with the response of the second antenna that forms the visibility. Three data outputs are necessary from gridding in order to calculate the based power spectrum with accurate error bars: the measured visibilities, gridded with the holographic beam model\footnote{Note that the resulting image will be tapered by the average primary beam squared}; the weights, obtained by gridding model visibilities with all values set to 1 with the holographic beam model; and the variance, obtained by gridding model visibilities with all values set to 1 with the squared holographic beam model.

The FHD calibration and imaging pipeline for the MWA is run in two different modes for EoR observations. In the first mode, we deconvolve a deep observation with XXX hours on each field to generate a complete map of the foregrounds. This foreground model includes both compact and extended sources and diffuse structure such as the galaxy, and it is used as an input to the second FHD imaging mode. In the second mode we process all 1000 hours of EoR observations, but do not perform any deconvolution. Instead, we generate improved model visibilities using the full polarized direction-dependent antenna gains, and calculate the on-field direction-independent calibration solutions for each 112s EoR snapshot. The residual visibilities formed by subtracting the calibration model from the data are first saved and sent to the visibility-based power spectrum pipeline (section XXX), and then gridded along with the weights and variance at a coarse resolution but fine frequency bandwidth for the image-based power spectrum pipeline (section XXX). For a long integration, the snapshot image, weights, and variance cubes are re-gridded to the HEALPix frame and stacked. Corrections for ionospheric distortions can be applied in the final re-gridding.
