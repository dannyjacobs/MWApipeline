\subsection{Imager \#1: RTS}
The MWA Real Time System (RTS) is a radio interferometry software package specifically written to calibrate and image MWA data \citep[][Mitchell et al. in prep]{dale_Cappallo_Morales_Ord_2008}. The RTS incorporates algorithms intended to address a number of known challenges inherent to processing MWA data, including; wide-field imaging effects, direction-dependant (DD) antenna gains and polarization response, and ionospheric refraction of low-frequency radio waves. Each MWA observation (112s) is processed through a separate instance of the RTS. The RTS is also parallelised over frequency so that each coarse channel (32 * 40 kHz fine channels) is processed largely independently of the other coarse channels, with only information about the measured ionospheric offsets communicated between processing nodes.  

The RTS calibration strategy is based upon the 'Peeling' technique proposed by \citet{Noordam_2004}. The brightest (apparent) radio sources in the field of view are sequentially and iteratively processed through a Calibrator Measurement Loop (CML). During each pass through the CML; i) the expected (model) visibilities of known catalogue sources are subtracted from the observed visibilities [reference to MWA catalog]. This step is called pre-peeling. For the data processed in this work, TBD sources are pre-peeled for each observation. ii) The model visibilites for the source being fitted are added back in and the peeled visibilities are rotated to have their phase center at the expected position of the source. Any ionospheric offset of the source can now be measured by fitting a phase ramp to the phased visibilities. iii) The strongest sources are now used to update the direction-dependant antenna gain terms, while weaker sources are only corrected for ionospheric offsets. For this work, TBD sources are used as full DD calibrators and TBD sources are set as ionospheric calibrators. The CML is repeated until the gain and ionospheric fits converge [define]. The TDB strongest sources are then subtracted from the calibrated visibilities and the residuals are passed to the visibility-based power spectrum described in section foo.

[Discuss time/frequency averaging, short baseline taper?]

[Numbers regarding ionopsheric offsets?]

[Bandpass Cal?]

The RTS uses a snapshot imaging approach to correct for wide-field and direction-dependant polarisation effects. Following calibration, the residual visibilities are first gridded to form instrumental polarization images which are co-planar with the array. These images are then regridded into the HEALPIX frame with wide-field corrections and conversion to Stokes applied through the regridding weights. It is also possible to use the fitted ionospheric calibrator offsets to apply a correction for ionospheric effects across the field during the regridding step, but in this work this correction has not been applied. See \citet{Clark_Allen_Arcus_et_al__2010} for a more complete description of the RTS imaging algorithms.

[Are any details of interface to UW PS necessary?]