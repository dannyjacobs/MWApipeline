\subsection{Power Spectrum #2: Visibility-based}
The second power spectrum estimation method computes the maximum likelihood estimate of the 21~cm power, using residual visibilities from the RTS and FHD. It aims to extract maximal information about the signal of interest by incorporating maximal information about the instrument and foreground signals by correctly characterizing the statistical properties of the data. This approach also allows straightforward estimation of the variances and covariances between sky modes by direct propagation of errors, and operates entirely within $uv$-space with complex-valued calibrated and peeled visibilities.

The method involves three major steps: (1) Grid and weight measured visibilities onto the $uv$-plane using the primary beam model, and combine all data into $w$-stacks; (2) compute the maximum-likelihood estimate of the cross-power spectrum, incorporating foregrounds and radiometric noise; (3) compute the uncertainties and covariances between power estimates. The first step is the most computationally-intensive, requiring processing of all the measured data. The second and third steps are performed iteratively on the final full set of beam-weighted visibilities. These steps can be performed multiple times using different foreground models.


A two paragraph description of how the power spectrum is estimated from residual visibilities.  How are error bars estimated (the math part). But also the operational part, which axes are multiplied, which are averaged etc.  

Cath todo