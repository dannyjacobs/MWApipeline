\subsection{Power Spectrum #2: Visibility-based}
The second power spectrum estimation method computes the maximum likelihood estimate of the 21~cm power, using residual visibilities from the RTS and FHD. It aims to extract maximal information about the signal of interest by incorporating maximal information about the instrument and foreground signals by correctly characterizing the statistical properties of the data. The approach is similar to that used by Liu & Tegmark (2011)[http://dx.doi.org/10.1103/physrevd.83.103006], but with the key difference of being performed entirely in $uv$-space, where the data covariance matrix is simpler (block diagonal), and feasible to invert. This approach also allows straightforward estimation of the variances and covariances between sky modes by direct propagation of errors, and operates entirely within $uv$-space with complex-valued calibrated and peeled visibilities.

The method involves four major steps: (1) Grid and weight measured visibilities onto the $uv$-plane using the primary beam model, and combine all data into $w$-stacks, for each spectral channel; (2) compute the least squares spectral (LSS) transform along the frequency dimension to obtain the best estimate of the line-of-sight spatial sky modes (this technique is comparable to that used in the image-based pipeline, described above); (3) compute the maximum-likelihood estimate of the cross-power spectrum, incorporating foregrounds and radiometric noise, by averaging $k_x$ and $k_y$ modes into annular modes on the sky, $k_\bot$; (4) compute the uncertainties and covariances between power estimates. The first step is the most computationally-intensive, requiring processing of all the measured data. It also includes a step to account for the spectral smearing of the sky information that occurs when observed through the instruments primary beam. This accounting aims to connect the information available from the sky with the information in the data. Unlike the image-based power spectrum method described above, the visibility-based pipeline uses a finely-sampled $uv$-plane to correctly characterise the smearing due to the primary beam.

The line-of-sight transform is performed once, and the uncertainties propagated. The final two steps in the pipeline are performed on the final full set of beam-weighted and transformed visibilities. These steps can be performed multiple times using different foreground models, and are also applied iteratively to converge to the maximum likelihood solution. The resulting dataset provides an estimate of the power in each $k_\bot,k_\parallel$ mode, and the variances and covariances associated with these estimates.

A two paragraph description of how the power spectrum is estimated from residual visibilities.  How are error bars estimated (the math part). But also the operational part, which axes are multiplied, which are averaged etc.  

Cath todo