\subsection{Power Spectrum \#1: Image-based}
The direct propagation power spectrum method calculates a straight-forward power spectrum estimate and directly propagates the error bars through the full analysis. It requires pairs of holographically gridded image cubes in which the data has been split into interleaved time samples (referred to as even and odd cubes) along with matched cubes containing the weights and the variances (Do we want to add equations to describe these?).  The data, weight and variance cubes are Fourier transformed in two dimensions to take them to uvf space where the covariance matrix is assumed in this method to be diagonal\footnote{Note that this is a better assumption if the uv pixel size is well matched to beam size}. The data (variance) cubes are then divided by the weight cubes (weight cubes squared) to arrive at the best estimates of the sky and variances. Next the sum and difference of the even and odd cubes are computed with variances given by adding the reciprocal of the even and odd variances in quadrature. The difference cube then contains only noise (as long as the time interleaving is fine enough) and the sum cube contains both sky signal and noise.

The next step is to apply a spectral window function (typically a Blackman-Harris) and to Fourier transform in the frequency direction. The covariance matrix (diagonal in uvf; terms given by the sum/difference variances) is propagated through this Fourier transform and covariances between non-identical modes are marginalized over. This results in a two-by-two covariance matrix for each Fourier mode containing the $cos^2$ and $sin^2$ terms on the diagonal and the $cos*sin$ cross term in the off-diagonal elements. These covariance matrices (and the associated data) are then diagonalized for each mode (as in Lomb (1976) [N.R. Lomb, Astrophysics and Space Science, vol 39, pp. 447-462, 1976] & Scargle (1982) [J.D. Scargle, The Astrophysical Journal, vol 263, pp. 835-853, 1982]), producing two terms that are squared and combined to get to power estimates. The square of the difference cube is a realization of the noise in the power spectrum and the sky signal power is best estimated by the square of the sum cube minus the square of the difference cube\footnote{This is mathematically identical to the even/odd cross power if the even and odd variances are identical}. Finally the power cubes can be averaged (using a variance weighting) in $k_x-k_y$ rings to get to a two dimensional $k_{\|}-k_{\bot}$ space.


A two paragraph description of how the image-based power spectrum estimation works. Focus on how the eta fourier transform is done, how error bars are estimated, cross-multiplication and averaging, and show an example from Aug 23 or 26.

bryna todo